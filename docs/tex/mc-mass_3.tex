\chapter{Mass Characterics}

It is assumed that all aircraft components are uniform rectangular cuboids, parallel to the aircraft Body Axis System, with position defined by their center of mass.

All values are expressed in International System of Units (SI).

\section{Center of Mass}

Aircraft center of mass position is given by the following formula. \cite{Taylor2005}
\begin{equation}
  \label{eq-center-of-mass}
  \vec{r}_{CG}
  =
  \frac{\sum_{j} m_j * \vec{r}_j}{\sum_{j} m_j}
\end{equation}

Where:
\begin{description}[align=right,labelwidth=1cm]
  \item [$m_j$]       --- [kg] component mass
  \item [$\vec{r}_j$] --- [m] component center of mass position
\end{description}

\section{Moment of Inertia}

Uniform rectangular cuboid moment of inertia about its center of mass is given by the following formula. \cite{Awrejcewicz2012}
\begin{equation}
  \label{eq-cuboid-inertia}
  {\boldsymbol I}_0
  =
  m
  \left[
    \begin{matrix}
      \frac{w^2 + h^2}{12} & 0 & 0 \\
      0 & \frac{l^2 + h^2}{12} & 0 \\
      0 & 0 & \frac{l^2 + w^2}{12} \\
    \end{matrix}
  \right]
\end{equation}

Where:
\begin{description}[align=right,labelwidth=1cm]
  \item [$m$] --- [kg] component mass
  \item [$l$] --- [m] component length (dimension along x-axis)
  \item [$w$] --- [m] component width (dimension along y-axis)
  \item [$h$] --- [m] component height (dimension along z-axis)
\end{description}

\section{Huygens–Steiner Parallel Axis Theorem}

Huygens–Steiner parallel axis theorem, given by the following expression, is used to calculate aircraft components inertia tensors expressed in the aircraft Body Axis System. \cite{Awrejcewicz2012}
\begin{equation}
  \label{eq-mass-steiners}
  {\boldsymbol I}_b
  =
  {\boldsymbol I}_0
  +
  m
  \left[
    \begin{matrix}
      y^2 + z^2 &       -xy &       -xz \\
            -yx & x^2 + z^2 &       -yz \\
            -zx &       -zy & x^2 + y^2 \\
    \end{matrix}
  \right]
\end{equation}

Where:
\begin{description}[align=right,labelwidth=1cm]
  \item [${\boldsymbol I}_0$] --- [kg$\cdot$m\textsuperscript{2}] component inertia about its center of mass
  \item [$m$] --- [kg] component mass
  \item [$x$] --- [m] component center of mass x-coordinate
  \item [$y$] --- [m] component center of mass y-coordinate
  \item [$z$] --- [m] component center of mass z-coordinate
\end{description}

Sum of all aircraft components inertia tensors gives aircraft inertia tensor:
\begin{equation}
  {\boldsymbol I}_{b} = \sum_{j} {\boldsymbol I}_{b,j}
\end{equation}

\section{Statistical Weight Estimation}

Lorem ipsum. \cite{NASA-TP-2015-218751,Raymer2018}

\subsection{Fighter/Attack}

\subsubsection{Wing}

Lorem ipsum.

\subsubsection{Horizontal Tail}

Lorem ipsum.

\subsubsection{Vertical Tail}

Lorem ipsum.

\subsubsection{Fuselage}

Lorem ipsum.

\subsubsection{Main Landing Gear}

Lorem ipsum.

\subsubsection{Nose Landing Gear}

Lorem ipsum.

\subsubsection{"All-else empty"}

Lorem ipsum.

\subsection{Cargo/Transport}

\subsubsection{Wing}

Lorem ipsum.

\subsubsection{Horizontal Tail}

Lorem ipsum.

\subsubsection{Vertical Tail}

Lorem ipsum.

\subsubsection{Fuselage}

Lorem ipsum.

\subsubsection{Main Landing Gear}

Lorem ipsum.

\subsubsection{Nose Landing Gear}

Lorem ipsum.

\subsubsection{"All-else empty"}

Lorem ipsum.

\subsection{General Aviation}

\subsubsection{Wing}

Lorem ipsum.

\subsubsection{Horizontal Tail}

Lorem ipsum.

\subsubsection{Vertical Tail}

Lorem ipsum.

\subsubsection{Fuselage}

Lorem ipsum.

\subsubsection{Main Landing Gear}

Lorem ipsum.

\subsubsection{Nose Landing Gear}

Lorem ipsum.

\subsubsection{"All-else empty"}

Lorem ipsum.

\subsection{Helicopter}

Lorem ipsum. \cite{NASA-TP-2015-218751}

\subsubsection{Fuselage}

Lorem ipsum.

\subsubsection{Landing Gear}

Lorem ipsum.

\subsubsection{Main Rotor}

Lorem ipsum.

\subsubsection{Tail Rotor}

Lorem ipsum.

\subsubsection{Rotor Drive System}

Lorem ipsum.

\subsubsection{Main Rotor Hub}

Lorem ipsum.

\subsubsection{Horizontal Tail}

Lorem ipsum.

\subsubsection{Vertical Tail}

Lorem ipsum.

\subsubsection{"All-else empty"}

Lorem ipsum.
