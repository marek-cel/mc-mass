\chapter{Mass Characterics}

It is assumed that all aircraft components are uniform rectangular cuboids, parallel to the aircraft Body Axis System, with position defined by their center of mass.

\section{Moment of Inertia}

Uniform rectangular cuboid moment of inertia about its center of mass is given by the following formula. \cite{Awrejcewicz2012}

\begin{equation}
  \label{eq-cuboid-inertia}
  {\boldsymbol I}
  =
  m
  \left[
    \begin{matrix}
      \frac{w^2 + h^2}{12} & 0 & 0 \\
      0 & \frac{l^2 + h^2}{12} & 0 \\
      0 & 0 & \frac{l^2 + w^2}{12} \\
    \end{matrix}
  \right]
\end{equation}

\section{Huygens–Steiner Parallel Axis Theorem}

Huygens–Steiner parallel axis theorem, given by the following expression, is used to calculate aircraft components inertia tensors. \cite{Awrejcewicz2012}
\begin{equation}
  \label{eq-mass-steiners}
  {\boldsymbol I}_b
  =
  {\boldsymbol I}_0
  +
  m
  \left[
    \begin{matrix}
      y^2 + z^2 &       -xy &       -xz \\
            -yx & x^2 + z^2 &       -yz \\
            -zx &       -zy & x^2 + y^2 \\
    \end{matrix}
  \right]
\end{equation}

Sum of all aircraft components inertia tensors gives aircraft inertia tensor:
\begin{equation}
  {\boldsymbol I}_b = \sum_{j} {\boldsymbol I}_{j,b}
\end{equation}

\section{Statistical Weight Estimation}

Lorem ipsum. \cite{NASA-TP-2015-218751,Raymer1992}

\subsection{Fighter/Attack}

\subsection{Cargo/Transport}

\subsection{General Aviation}

\subsection{Helicopter}

Lorem ipsum. \cite{NASA-TP-2015-218751}
